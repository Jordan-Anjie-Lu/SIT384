\documentclass{article}

\usepackage{amsmath}
\usepackage{minted}
\usepackage{hyperref}

\title{Recursive Programs for Counting}
\date{2020-05-30}


\begin{document}
	\pagenumbering{gobble}
	\maketitle
	\newpage
	\pagenumbering{arabic}
	\tableofcontents
	\newpage
	
	\section{Explanation}
	
	(LaTeX)
	When completing the Counting unit, I had just completed some coding in Python for another unit
	(SIT384, Cyber Security Analytics). I'd completed a recursive factorial script earlier in the trimester
	for some fun, and I had still had my terminal open. I used the script to give me factorial values, then
	decided to quickly write up some scripts to do combination, permutation, binomials and combination repetition
	(later shortened to just combination, permutation and combination repetition as binomials are combinations).
	
	\section{Repetition}
	
	All of the following code utilises repetition to derive the factorial values.
	That code is provided here:
	
	\begin{minted}{python}
def recursion(input):
	if input > 1:
		result = input * recursion(input-1)
	else:
		result = 1
	return result
	\end{minted}
	
	This code stops running at 999 (producing values up to and including 998) due to
	how Python does repetition. As it is, you shouldn't really need values past 1000.
	
	Future versions may include hard-coded values at each 20th interval (i.e. 20,40...)
	to speed up processing.
	
	\section{Factorial}
	
	The initial code for factorial calculation is included for posterity, as all scripts used originate from here.
	It takes a value and checks if it's larger than 0. If larger than 0,
	it conducts recursion and prints the result. If smaller, it requests a value that is positive.
	
	\begin{minted}{python}
while True:
	accepted = int(input("Please input a non-negative integer: "))
	if accepted >= 0:
		print(recursion(accepted))
	else:
		print("Please input a number that has no negative.")
	\end{minted}
	
	\section{Combination}
	
	Combination is utilised when you need the number of all possible
	results from a given selection. This is usually displayed as the following equation:

	\begin{equation*}
	C(n,k) = \frac{n!}{k!(n-k)!}
	\end{equation*}
	
	For example, if you needed all the 3-letter combinations from 5 letters, the
	equation required would be:
	
	\begin{align*}
	C(5,3) & = \frac{5!}{3!(5-3)!}\\
	& = \frac{5!}{3!2!}\\
	& = 10
	\end{align*}
	
	The Python code written below takes the values n and k, and produces
	the result. It utilises a while statement to allow for continuous input.
	
	\paragraph{}
	It works by taking the two inputs (n and k), making n-k as a separate variable,
	obtaining the recursive values of all three then conducting the equation and printing
	the result.
	
	
	\begin{minted}{python}
while True:
	result = 1
	val1 = int(input("Value one:"))
	val2 = int(input("Value two:"))
	if val1 >= 0:
		val3 = val1 - val2
		val1 = recursion(val1)
		val2 = recursion(val2)
		val3 = recursion(val3)
		val4 = val1 / (val3 * val2)
		print(val4)
	else:
		print("Please input a number that has no negative.")
	\end{minted}
	
	The code can be compressed by writing the recursion into the equation, instead of doing it separately.
	However, for code clarity, it has been separated.
	
	\section{Permutation}
	
	Permutation is utilised when you the number of possible results
	where the output follows a certain order.
	This is usually displayed as the following equation:

	\begin{equation*}
	P(n,k) = \frac{n!}{(n-k)!}
	\end{equation*}
	
	For example, if you needed all the 3-letter combinations from 5 letters in alphabetical order, the
	equation required would be:
	
	\begin{align*}
	P(5,3) & = \frac{5!}{(5-3)!}\\
	& = \frac{5!}{2!}\\
	& = 60
	\end{align*}
	
	The Python code written below takes the values n and k, and produces
	the result. It utilises a while statement to allow for continuous input.
	
	\paragraph{}
	It works by taking the two inputs (n and k), making n-k in place, then obtaining
	the recursive values of both and dividing them, before printing the result.
	
	\begin{minted}{python}
while True:
	result = 1
	val1 = int(input("Value one:"))
	val2 = int(input("Value two:"))
	if val1 >= 0:
		val2 = val1 - val2
		val1 = recursion(val1)
		val2 = recursion(val2)
		val3 = val1 / val2
		print(val3)
	else:
		print("Please input a number that has no negative.")
	\end{minted}
	
	The code can be compressed by writing the recursion into the equation, instead of doing it separately.
	However, for code clarity, it has been separated.
	
	
	\section{Combination Repetition}
	
	Combination repetition takes a set A, and tries to stuff it in as many different formats as possible.
	For example, given a set $S=\{A,B,C,D\}$ and asked the question "How many times can you make a different 3-letter value?",
	you can produce "AAA, AAB, AAC, AAD, ABB, ABC, ABD, ACC, ACD, ADD, BBB, BBC, BBD, BCC, BCD, BDD, CCC, CCD, CDD, DDD.", which is 15 values.

	\paragraph{}
	That description is bad. Let's just use the equation (which uses a non-standard designation).
	
	\begin{equation*}
	CR(n,k) = \frac{(n+k-1)!}{k!(n-1)!}
	\end{equation*}
	
	When given the same example as the rest (5 letters, 3-letter output) you get the following result:
	
	\begin{align*}
	CR(5,3) & = \frac{(5+3-1)!}{3!(5-1)!}\\
	& = \frac{7!}{3!4!} \\
	& = 35
	\end{align*}
	
	The Python code written below takes the values n and k, and produces
	the result. It utilises a while statement to allow for continuous input.
	
	\paragraph{}
	It works by taking the two inputs (n and k), making n+k-1 as a separate variable, making n-1 from n+k-1 as another variable,
	then obtaining the recursive values of all four and dividing with the equation, before printing the result.
	Making n-1 from n+k-1 was blind luck, as this code is originally a modification of the combination code.
	
	\begin{minted}{python}
while True:
	result = 1
	val1 = int(input("Value one:"))
	val2 = int(input("Value two:"))
	if val1 >= 0:
		val1 = val1 + val2 - 1
		val3 = val1 - val2
		val1 = recursion(val1)
		val2 = recursion(val2)
		val3 = recursion(val3)
		val4 = val1 / (val3 * val2)
		print(val4)
	else:
		print("Please input a number that has no negative.")
	\end{minted}
	
	The code can be compressed by writing the recursion into the equation, instead of doing it separately.
	However, for code clarity, it has been separated.	
	
	\section{Tools used to write this document}
	\paragraph{Notepad++}
	using the \href{https://draculatheme.com/notepad-plus-plus/}{Dracula} theme
	with ligatures enabled. (URL: \url{https://draculatheme.com/notepad-plus-plus/})
	\paragraph{MikTeX}
	using PDFTex+MakeIndex+BibTeX with the amsmath, minted and hyperref packages.
	(URL: \url{https://miktex.org/})
	\paragraph{Internet}
	with far too much browsing for information for my liking.
	

\end{document}